% -*- coding: utf-8 -*-
%-------------------------
% Resume in Latex
% Author : Jing Wan
% Based off of: https://github.com/sb2nov/resume
% License : MIT
%------------------------

\documentclass[a4paper,11pt]{article}

\usepackage{latexsym}
\usepackage[empty]{fullpage}
\usepackage{titlesec}
\usepackage{marvosym}
\usepackage[usenames,dvipsnames]{color}
\usepackage{verbatim}
\usepackage{enumitem}
\usepackage[hidelinks]{hyperref}
\usepackage{fancyhdr}
\usepackage[english]{babel}
\usepackage{tabularx}
%\usepackage[UTF8]{ctex} %for latex
\usepackage{fontspec}
\usepackage{xeCJK}
\setCJKmainfont{Source Han Serif CN}
%\input{glyphtounicode} %for latex
\XeTeXgenerateactualtext=1

%----------FONT OPTIONS----------
% sans-serif
% \usepackage[sfdefault]{FiraSans}
% \usepackage[sfdefault]{roboto}
% \usepackage[sfdefault]{noto-sans}
% \usepackage[default]{sourcesanspro}

% serif
% \usepackage{CormorantGaramond}
% \usepackage{charter}


\pagestyle{fancy}
\fancyhf{} % clear all header and footer fields
\fancyfoot{}
\renewcommand{\headrulewidth}{0pt}
\renewcommand{\footrulewidth}{0pt}

% Adjust margins
\addtolength{\oddsidemargin}{-0.5in}
\addtolength{\evensidemargin}{-0.5in}
\addtolength{\textwidth}{1in}
\addtolength{\topmargin}{-.5in}
\addtolength{\textheight}{1.0in}

\urlstyle{same}

\raggedbottom
\raggedright
\setlength{\tabcolsep}{0in}

% Sections formatting
\titleformat{\section}{
  \vspace{-4pt}\scshape\raggedright\large
}{}{0em}{}[\color{black}\titlerule \vspace{-5pt}]

% Ensure that generate pdf is machine readable/ATS parsable
%\pdfgentounicode=1 %for latex

%-------------------------
% Custom commands
\newcommand{\resumeItem}[1]{
  \item\small{
    {#1 \vspace{-2pt}}
  }
}

\newcommand{\resumeSubheading}[4]{
  \vspace{-2pt}\item
    \begin{tabular*}{0.97\textwidth}[t]{l@{\extracolsep{\fill}}r}
      \textbf{#1} & #2 \\
      \textit{\small#3} & \textit{\small #4} \\
    \end{tabular*}\vspace{-7pt}
}

\newcommand{\resumeSubSubheading}[2]{
    \item
    \begin{tabular*}{0.97\textwidth}{l@{\extracolsep{\fill}}r}
      \textit{\small#1} & \textit{\small #2} \\
    \end{tabular*}\vspace{-7pt}
}

\newcommand{\resumeProjectHeading}[2]{
    \item
    \begin{tabular*}{0.97\textwidth}{l@{\extracolsep{\fill}}r}
      \small#1 & #2 \\
    \end{tabular*}\vspace{-7pt}
}

\newcommand{\resumeSubItem}[1]{\resumeItem{#1}\vspace{-4pt}}

\renewcommand\labelitemii{$\vcenter{\hbox{\tiny$\bullet$}}$}

\newcommand{\resumeSubHeadingListStart}{\begin{itemize}[leftmargin=0.15in, label={}]}
\newcommand{\resumeSubHeadingListEnd}{\end{itemize}}
\newcommand{\resumeItemListStart}{\begin{itemize}}
\newcommand{\resumeItemListEnd}{\end{itemize}\vspace{-5pt}}

\setlength{\footskip}{5pt}

%-------------------------------------------
%%%%%%  RESUME STARTS HERE  %%%%%%%%%%%%%%%%%%%%%%%%%%%%


\begin{document}

%----------HEADING----------
% \begin{tabular*}{\textwidth}{l@{\extracolsep{\fill}}r}
%   \textbf{\href{http://sourabhbajaj.com/}{\Large Sourabh Bajaj}} & Email : \href{mailto:sourabh@sourabhbajaj.com}{sourabh@sourabhbajaj.com}\\
%   \href{http://sourabhbajaj.com/}{http://www.sourabhbajaj.com} & Mobile : +1-123-456-7890 \\
% \end{tabular*}

\begin{center}
    \textbf{\Huge \scshape 万晶} \\ \vspace{1pt}
    \small 156-0198-1011 $|$ \href{mailto:wanjingjjj@gmail.com}{\underline{wanjingjjj@gmail.com}} $|$ 
    \href{https://github.com/wanjingjjj}{\underline{github.com/wanjingjjj}}
\end{center}


%-----------EDUCATION-----------
\section{教育背景}
  \resumeSubHeadingListStart
    \resumeSubheading
      {上海交通大学}{上海}
      {生物技术理学学士}{2003年9月 -- 2007年6月}
  \resumeSubHeadingListEnd


%-----------EXPERIENCE-----------
\section{工作经历}
  \resumeSubHeadingListStart

    \resumeSubheading
      {Adevinta (ebay classifieds group)}{上海}
      {资深软件工程师(L26)}{2020年11月 -- 现在}
      \resumeItemListStart
      \resumeItem{设计开发基于Airflow和Spark的数据平台,使用云原生技术运行于\emph{EKS}。}
      \resumeItem{开发维护内部的Spark版本,基于开源社区项目,在此基础上添加了对\emph{AWS} glue, delta.io, Arm架构和云原生的支持。}
      \resumeItem{基于开源Jupyter和VSCode开发内部的Notebook工具,用于数据分析与机器学习开发。前端基于\emph{React}后端基于\emph{go}。}
      \resumeItem{定期复查与优化\emph{AWS}资源使用成本,保证资源能够充分合理的使用。}
       \resumeItemListEnd
      
% -----------Multiple Positions Heading-----------
%    \resumeSubSubheading
%     {Software Engineer I}{Oct 2014 - Sep 2016}
%     \resumeItemListStart
%        \resumeItem{Apache Beam}
%          {Apache Beam is a unified model for defining both batch and streaming data-parallel processing pipelines}
%     \resumeItemListEnd
%    \resumeSubHeadingListEnd
%-------------------------------------------

    \resumeSubheading
      {氪信科技}{上海}
      {数据开发部经理}{2018年9月 -- 2020年10月}
      \resumeItemListStart
        \resumeItem{带领团队开发数据产品,主要服务于银行客户。}
        \resumeItem{客户自建数据中心架构上的数据产品部署与维护。} 
        \resumeItem{沟通与协调多个实施项目的人员与技术方案。}
      \resumeItemListEnd

      \resumeSubSubheading
          {高级数据工程师}{2015年6月 -- 2018年8月}
          \resumeItemListStart
          \resumeItem{在 \emph{AWS} 云上从头搭建基于基于\emph{Airflow} 和 \emph{CDH} 的数据仓库平台。}
          \resumeItem{维护开发统一日志采集服务。}
          \resumeItem{创建基于 \emph{Superset} 的BI报表服务。销售与分析人员可自助式浏览与创建数据报表。}
          \resumeItem{为机器学习模型处理数据与开发特征指标。}
          \resumeItemListEnd

    \resumeSubheading
      {ebay}{上海}
      {高级软件工程师(L24)}{2011年3月 -- 2015年5月}
      \resumeItemListStart
        \resumeItem{使用 \emph{UC4(Appworx)} 和 \emph{SAS} 开发维护账务数据分析批处理作业。}
        \resumeItem{开发内部自助工具以帮助销售与客服更方便的获取账务数据。}
        \resumeItem{自动化某些账户流程。}
      \resumeItemListEnd
      
    \resumeSubheading
      {阿里巴巴集团}{杭州}
      {ETL工程师}{2010年9月 -- 2011年3月}
      \resumeItemListStart
        \resumeItem{开发维护\emph{阿里金融}数据仓库平台。}
        \resumeItem{为在线小微贷款机器学习模型创建数据流程,开发数据指标特征。}
      \resumeItemListEnd

    \resumeSubheading
      {ebay (诚归计算机)}{上海}
      {数据仓库工程师}{2007年7月 -- 2010年9月}
      \resumeItemListStart
        \resumeItem{在 \emph{ebay Teradata} 平台上开发 \emph{ETL} 流程。}
        \resumeItem{数据批处理作业的 \emph{SQL} 调优与问题解决。}
      \resumeItemListEnd

  \resumeSubHeadingListEnd

%-----------PROJECTS-----------
\section{项目}
    \resumeSubHeadingListStart
      \resumeProjectHeading
          {\textbf{Notebook 作业调度} $|$ \emph{AWS(EKS), Kubernetes, Argo Workflow, React, Go}}{2023年7月 -- 2024年6月}
          \resumeItemListStart
            \resumeItem{使用 \emph{Argo Workflow} 定时调度 \emph{jupyter notebook} 文件。}
            \resumeItem{使用 \emph{kubernetes operator} 存储与管理调度定义。}
            \resumeItem{基于 \emph{React} 的前端网页界面供用户管理所有的调度任务。}
          \resumeItemListEnd
      \resumeProjectHeading
          {\textbf{数据仓库上云} $|$ \emph{AWS(EKS, S3, Glue), Airflow, Spark, Kubernetes, Python}}{2021年6月 -- 2023年6月}
          \resumeItemListStart
            \resumeItem{使用 \emph{EKS} 做为基础架构, \emph{Airflow} 做为流程编排引擎, \emph{Spark} 做为数据处理引擎。所有的服务都为云原生。}
            \resumeItem{创建包含所有工具链与ETL框架的 \emph{Docker} 镜像,提供所有 \emph{ETL} 作业的运行时。}
            \resumeItem{创建包含补丁可以 \emph{AWS Glue} 做为 Hive 元数据的 \emph{Spark} 镜像。}
            \resumeItem{从现有 \emph{ebay} 数据平台迁移超过100个数据流程到新平台。}
            \resumeItem{为新平台设计与创建持续集成与持续发布流程。}
          \resumeItemListEnd
      \resumeProjectHeading
          {\textbf{AI 数据服务} $|$ \emph{Python, Go, CDH(Hive), Java}}{2016年6月 -- 2018年8月}
          \resumeItemListStart
            \resumeItem{使用 \emph{Python} 和 \emph{Go} 创建后端数据服务,供在线机器学习模型调用。}
            \resumeItem{统一数据采集服务将机器学习模型的中间过程与最终结果保存到 \emph{Hadoop}。}
            \resumeItem{使用 \emph{Hive} 将采集的日志数据与其他数据做处理,供分析模型迭代使用。}
          \resumeItemListEnd
    \resumeSubHeadingListEnd


%
%-----------PROGRAMMING SKILLS-----------
\section{技术栈}
 \begin{itemize}[leftmargin=0.15in, label={}]
    \small{\item{
     \textbf{语言}{: Python, SQL (SparkSQL / Hive), Go, Java, JavaScript, HTML/CSS} \\
     \textbf{框架}{: Spark, Hadoop, Flask, pandas, Matplotlib, Airflow, Argoflow, React, Beego, Springboot } \\
     \textbf{开发工具}{: Git, Kubernetes (EKS), Docker, AWS, Terraform, Datadog, Grafana} \\
    }}
 \end{itemize}


%-------------------------------------------
\end{document}
